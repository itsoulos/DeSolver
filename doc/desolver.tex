%% LyX 1.3 created this file.  For more info, see http://www.lyx.org/.
%% Do not edit unless you really know what you are doing.
\documentclass{article}
\usepackage[]{fontenc}
\usepackage{float}
\usepackage{setspace}
\doublespacing

\makeatletter

%%%%%%%%%%%%%%%%%%%%%%%%%%%%%% LyX specific LaTeX commands.
%% Bold symbol macro for standard LaTeX users
\newcommand{\boldsymbol}[1]{\mbox{\boldmath $#1$}}


%%%%%%%%%%%%%%%%%%%%%%%%%%%%%% Textclass specific LaTeX commands.
 \newenvironment{lyxcode}
   {\begin{list}{}{
     \setlength{\rightmargin}{\leftmargin}
     \setlength{\listparindent}{0pt}% needed for AMS classes
     \raggedright
     \setlength{\itemsep}{0pt}
     \setlength{\parsep}{0pt}
     \normalfont\ttfamily}%
    \item[]}
   {\end{list}}

%%%%%%%%%%%%%%%%%%%%%%%%%%%%%% User specified LaTeX commands.
\usepackage{listings}
\def\lstlistingname{�����}
        \def\lstlistlistingname{�����������}
        \lstset{language=C++,
%               backgroundcolor=\color{lightgray},
                extendedchars=true,
                showstringspaces=false,
                stringstyle=\ttfamily,
             %   caption=[����� \thelstlisting],
              %  title=����� \thelstlisting,
                frame=lines,
 emptylines=0
                }

\makeatother
\begin{document}


\title{DeSolver: solving differential equations with grammatical evolution}


\author{Ioannis G. Tsoulos I.E. Lagaris%
\footnote{Corresponding author: lagaris@cs.uoi.gr%
}}


\date{Department of Computer Science, University of Ioannina\\
P.O. Box 1186, Ioannina 45110 - GREECE}

\maketitle
\begin{abstract}
DeSolver is a tool for solving ordinary and partial differential equations
based on grammatical evolution. It is written entirely in ANSI C++,
but the user can develop his part either in C++ or in Fortran 77.
The solutions produces are in closed analytical form. We describe
the tool and we give some examples.
\end{abstract}
\textbf{PACS}: 02.30.Hq, 02.30.Jr, 02.60.Lj, 02.60.Pn


\section*{PROGRAM SUMMARY}

\begin{flushleft}\textit{Title of program}: DESOLVER\end{flushleft}

\begin{flushleft}\textit{Catalogue identifier}:\end{flushleft}

\begin{flushleft}\textit{Program available from}: CPC Program Library,
Queen's University of Belfast, N. Ireland.\end{flushleft}

\begin{flushleft}\textit{Computer for which the program is designed
and others on which it has been tested}: DESOLVER is designed for
UNIX machines. \end{flushleft}

\begin{flushleft}\emph{Installation}: University of Ioannina, Greece.\end{flushleft}

\begin{flushleft}\textit{Programming language used}: GNU-C++, GNU
Fortran - 77.\end{flushleft}

\begin{flushleft}\emph{Memory required to execute with typical data}:
1Megabyte.\end{flushleft}

\begin{flushleft}\textit{No. of bits in a word}: 32\end{flushleft}

\begin{flushleft}\emph{No. of processors used}: 1\end{flushleft}

\begin{flushleft}\emph{Has the code been vectorised or parallelized?}:
No.\end{flushleft}

\begin{flushleft}\emph{No. of bytes in distributed program,including
test data etc}.: 160 Kbytes.\end{flushleft}

\begin{flushleft}\emph{Distribution format}: gzipped tar file.\end{flushleft}

\begin{flushleft}\emph{Keywords}: Ordinary differential equations,
Partial Differential equations, genetic programming, grammatical evolution,
optimization.\end{flushleft}

\begin{flushleft}\emph{Nature of physical problem}: Analytical solution
of ordinary differential equations, systems of them as well as partial
differential equations.\end{flushleft}

\begin{flushleft}\emph{Method of solution}: The system utilizes the
method of grammatical evolution.\end{flushleft}

\begin{flushleft}\emph{Typical running time}: Depending on the complexity
of the differential equation to be solved.\end{flushleft}


\section*{LONG WRITE UP}


\section{Introduction}

The aim of this tool is to recover closed form analytical solutions
of ordinary differential equations (ODEs), systems of differential
equations (SODEs) and partial differential equations (PDEs). The program
is based on a method developed by our group and is described in detail
in \cite{key-44}. It relies on grammatical evolution, which is an
evolutionary approach that produces valid code in accord to a set
of grammatical rules. Grammatical evolution is detailed in \cite{key-39,key-40,key-41}.
The tool is designed for UNIX systems equipped with the gnu C++ and
Fortran 77 (g77) compilers.

The rest of this article is organized as follows: in section 2 we
present the contents of the distribution file, in section 3 the set
of prerequisites for the installation of the tool is presented, in
section 4 we explain the installation procedure of the DeSolver, in
section 5 we present the programming libraries developed for the DeSolver
tool and the produced binary files, in section 6 we present the codification
scheme for the differential equations and in section 7 we list some
test runs from the application of the programming tool.


\section{Distribution}

The tool is distributed in a tar.gz file named \texttt{DeSolver.tar.gz},
a format which is recognized by almost all operating systems. In a
Unix system issuing the following commands:

\begin{enumerate}
\item gunzip \texttt{DeSolver.tar.gz}
\item tar xfv \texttt{DeSolver.tar}
\end{enumerate}
will create the directory \texttt{Desolver} with the following subdirectories:

\begin{enumerate}
\item \texttt{\textbf{bin}}: Directory where the binaries will be installed.
\item \texttt{\textbf{doc}}: Directory containing the user manual (this
file) in different formats such as \TeX{}, Postscript and PDF.
\item \texttt{\textbf{include}}: Directory where the include files and specifically
the files with the definition of the main classes of the tool will
be installed. 
\item \texttt{\textbf{lib}}: This directory is initially empty. It will
host the tool libraries following compilation.
\item \texttt{\textbf{libGT}}: Directory containing the \texttt{libgt} library
that arranges the grammatical evolution and the genetic operations.
\item \texttt{\textbf{libOde}}: Directory containing the library \texttt{libode},
used for the description of ODEs, SODEs and PDEs.
\item \texttt{\textbf{libParser}}: Directory containing the \texttt{libparser}
library, used for evaluating produced functions their derivatives.
\item \texttt{\textbf{OdeExamples}}: Directory containing examples of ODEs
in C++ and in Fortran, with the appropriate format as required by
the \texttt{libode} library.
\item \texttt{\textbf{PdeExamples}}: Directory containing examples of PDEs
in C++ and in Fortran, with the appropriate format as required by
the \texttt{libode} library.
\item \texttt{\textbf{SodeExamples}}: Directory containing examples of SODEs
in C++ and in Fortran, with the appropriate format as required by
the \texttt{libode} library.
\end{enumerate}
In the main \texttt{DeSolver} directory there are the following files:

\begin{enumerate}
\item \texttt{\textbf{INSTALL}}: Text file describing the installation steps.
\item \texttt{\textbf{compile.bash}}: Script, that compiles the DeSolver,
on UNIX systems running the bash shell.
\item \texttt{\textbf{variables.bash}}: Script to customize the .bashrc
file.
\item \texttt{\textbf{compile.csh}}:Script, that compiles the DeSolver,
on UNIX systems running the csh shell.
\item \texttt{\textbf{variables.csh}}: Script to customize the .cshrc file.
\item \texttt{\textbf{Makefile}}: It is the basic Makefile for the system. 
\item \texttt{\textbf{osname.pl}}: Small script in Perl , that prints the
name of the host operating system.
\end{enumerate}

\section{Prerequisites}

The following software packages are required in order to compile the
tool:

\begin{enumerate}
\item \textbf{gnu-C++}: The tool is written entirely in gnu C++. The \texttt{DeSolver}
tool can not be compiled with other C++ compilers.
\item \textbf{gnu-Fortran (g77)} : This is required only if the user wishes
to write his equations in Fortran.
\item \textbf{Perl}: To install \texttt{DeSolver} under UNIX, different
makefiles are required for different UNIX versions. This is facilitated
via a provided, modified version of the tmake \cite{key-42} tool
written in Perl. The user can compile the entire system without the
use of Perl and tmake, but this procedure requires additional programming
skills and effort.
\item One of \textbf{}\texttt{bash} or \textbf{}\texttt{\textbf{csh}}: All
UNIX systems have one or more programming shells. Among the most popular
are the \texttt{bash} and the \texttt{csh} shell. We include installation
scripts for both shells, but any user with a little experience in
shell programming can port the installation scripts to other shells
as well.
\end{enumerate}

\section{Installation}

The steps for installing the \texttt{DeSolver} tool are the following:

\begin{enumerate}
\item Uncompress the distribution file as described in section 2.
\item cd \texttt{DeSolver}.
\item If in bash shell issue the commands\\
\texttt{compile.bash}~\\
\texttt{variables.bash}
\item If in csh shell issue the commands\\
\texttt{compile.csh}\\
\texttt{variables.csh}
\end{enumerate}
The first command creates the appropriate Makefiles and builds the
entire system, while the second one adds the necessary lines to the
main configuration file of the corresponding shell. This file is the
\texttt{.bashrc} in the main user directory for the \texttt{bash}
shell and the file .\texttt{cshrc} in the main user directory for
the \texttt{csh} shell. The system has been successfully installed
on GNU Linux systems, Solaris, Freebsd, Openbsd and IRIX.


\section{Detailed description}


\subsection{The libgt library}

The \texttt{libgt} library consists of four classes and it is general
enough to adapt in every programming language given the BNF definition.
The first one is the GSymbol class, which represents the terminal
and non - terminal symbols of the BNF grammar. The second class is
the GRule class, which represents the production rules of the BNF
grammar. The third class is the GProgram class, which represents programs
in the given BNF grammar. Finally, the GPopulation class represents
a population of chromosomes. Each chromosome is an instance of the
GProgram class, with a problem depended fitness function. The crossover
is performed with one - point crossover and the selection of parents
for the crossover procedure is performed with the method of tournament
selection. 


\subsection{The libparser library}

The \texttt{libparser} is used for the evaluation of the produced
expressions and their first and second derivatives. The library contains
two classes: the GDoubleStack class and the FunctionParser class.
The first class is a stack of double values. The Standard Template
Library of GNU C++ is equipped with a template for stacks, however
we chose to write our own class, in order to accelerate the operations
of push and pop and to perform checks for the values prior to the
push operation. The second class is the FunctionParser library \cite{key-43},
with the addition of some methods for the evaluation of first and
second derivatives of the emerging expressions. 


\subsection{The libode library }

The third library is the \texttt{libode} program library. This library
contains three classes: the GOdeProgram, the GSodeProgram and the
GPdeProgram. Every class inherits the GProgram class and implements
the virtual fitness method of the GProgram class. The GOdeProgram
class represents trial solutions for ODEs, the GSodeProgram represents
trial solutions for SODEs and the GPdeProgram class represents trial
solutions for PDEs. In every class the user can pass as parameters
through the corresponding methods the elements of the differential
equation to be solved or he can pass as a parameter to the constructor
method the location of the differential equation in the file system.
The differential equation must be in shared library format and the
formulation of the equation for every case is presented in the following
section.


\subsection{Binary files}

We have created six driver programs for solving differential equations.
All the programs are located in the \texttt{bin} directory. The programs
\texttt{ode\_simple} and \texttt{ode\_cmd} are appropriate for solving
ODEs, the programs \texttt{sode\_simple} and \texttt{sode\_cmd} for
solving SODEs and the programs \texttt{pde\_simple} and \texttt{pde\_cmd}
for solving PDEs. The programs with the suffix \texttt{\textbf{simple}}
do not have any special command - line parameter. They accept only
an argument that specifies the location of the shared library hosting
the differential equation. These programs have default values for
the system parameters such as the size of the genetic population or
the selection rate and hence the user must edit the programs in order
to adjust them accordingly. On the other hand, the programs with the
suffix \texttt{\textbf{cmd}} are customizable and the user may specify
the following parameters in the command line:

\begin{enumerate}
\item -\textbf{h:} A help screen for the parameters is presented.
\item -\textbf{p path,} where path is the shared library hosting the differential
equation to be solved.
\item -\textbf{c count}, where count indicates the desired number of chromosomes
in the population. The default value is 1000.
\item -\textbf{l len}, where len is the size of each chromosome in the population.
The default size of the chromosomes is 50.
\item -\textbf{r} \textbf{sd}, where sd is the seed for the random number
generator . The default value is 1000. We use the function rand()
from the standard library of C++.
\item -\textbf{g} \textbf{generations}, where generations is the maximum
number of generations allowed. The default value for this parameter
is 2000.
\item -\textbf{e} \textbf{eps}, where eps is the threshold for the termination
of the genetic algorithm. If the best chromosome fitness falls below
that threshold, then the system stops creating generations. The default
value for this parameter is $10^{-7}$.
\item -\textbf{s} \textbf{srate}, where srate is used as the selection rate
of the population. The user can specify values in the range {[}0,1{]}.
The default value for this parameter is 0.1.
\item -\textbf{m} \textbf{mrate}, where mrate is used as the mutation rate
for the population. The user can specify values in the range {[}0,1{]}.
The default value for this parameter is 0.05.
\end{enumerate}

\section{User written subprograms}

The user can program his differential equations either in c++ or in
Fortran. In this section examples for each case (ODE, SODE and PDE)
are presented.


\subsection{Format for ODEs}

In figures \ref{c++-ode} and \ref{fortran-ode} we present the formulation
for ODEs in the languages c++ and Fortran correspondingly. The listed
functions have the following meaning:

\begin{enumerate}
\item \texttt{getx0()}: Returns the lower boundary point, $x_{0}$.
\item \texttt{getx1()}: Returns the upper boundary point, $x_{1}$.
\item \texttt{getkind()}: Returns 1,2 or 3. Code 1 indicates that the ODE
is of first order and the boundary condition is of the form: $y(x_{0})=y_{0}$.
Code 2 indicates that the ODE is of second order with boundary conditions
of the form: $y(x_{0})=y_{0},\  y'(x_{0})=y'_{0}$. Finally, code
3 indicates that the ODE is of second order with boundary conditions
of the form: $y(x_{0})=y_{0},\  y(x_{1})=y_{1}$.
\item \texttt{getnpoints()}: Returns the number of training points.
\item \texttt{getf0()}: Returns the boundary condition on the left, $y_{0}$.
\item \texttt{getf1()}: Returns the boundary condition on the right, $y_{1}$.
\item \texttt{getff0()}: Returns the left boundary condition for second
order ODEs $y'_{0}$.
\item \texttt{ode1ff(x,y,yy)}: If the ODE is of first order, then the purpose
of the tool is to minimize the function $\mbox{ode1ff}(x,M(x),M'(x))$,
for different values of $x$ in the range $[x_{0},x_{1}]$. The function
$M(x)$ is a function that arise from the procedure of grammatical
evolution. The parameter y represents $M(x)$ and the parameter yy
represents $M'(x)$. 
\item \texttt{ode2ff(x,y,yy,yyy)}: If the ODE is of second order, then the
\texttt{DeSolver} tries to minimize the function $\mbox{ode2ff}(x,M(x),M'(x),M''(x))$,
for different values of $x$ in the range $[x_{0},x_{1}].$ As for
\texttt{ode1ff} the function $M(x)$ emerges from the process of grammatical
evolution. The parameter y represents $M(x)$, the parameter yy represents
$M'(x)$ and the parameter yyy represents $M''(x)$.
\end{enumerate}
%
\begin{figure}[H]

\caption{Ode format in C++ \label{c++-ode}}

\texttt{\textbf{extern}} \texttt{\char`\"{}C\char`\"{}}

\texttt{\{}

\texttt{$\quad$}\texttt{\textbf{double}} \texttt{getx0()}

\texttt{$\quad$\{}

\texttt{$\quad$\}}~\\


\texttt{$\quad$}\texttt{\textbf{double}} \texttt{getx1()}

\texttt{$\quad$\{}

\texttt{$\quad$\}}~\\


\texttt{$\quad$}\texttt{\textbf{int}} \texttt{getkind()}

\texttt{$\quad$\{}

\texttt{$\quad$\}}~\\


\texttt{$\quad$}\texttt{\textbf{int}} \texttt{getnpoints()}

\texttt{$\quad$\{}

\texttt{$\quad$\}}~\\


\texttt{$\quad$}\texttt{\textbf{double}} \texttt{getf0()}

\texttt{$\quad$\{}

\texttt{$\quad$\}}~\\


\texttt{$\quad$}\texttt{\textbf{double}} \texttt{getf1()}

\texttt{$\quad$\{}

\texttt{$\quad$\}}~\\


\texttt{$\quad$}\texttt{\textbf{double}} \texttt{getff0()}

\texttt{$\quad$\{}

\texttt{$\quad$\}}~\\


\texttt{$\quad$}\texttt{\textbf{double}} \texttt{ode1ff(}\texttt{\textbf{double}}
\texttt{x,}\texttt{\textbf{double}} \texttt{y,}\texttt{\textbf{double}}
\texttt{yy)}

\texttt{$\quad$\{}

\texttt{$\quad$\}}~\\


\texttt{$\quad$}\texttt{\textbf{double}} \texttt{ode2ff(}\texttt{\textbf{double}}
\texttt{x,}\texttt{\textbf{double}} \texttt{y,}\texttt{\textbf{double}}
\texttt{yy,}\texttt{\textbf{double}} \texttt{yyy) }

\texttt{$\quad$\{}

\texttt{$\quad$\}}

\texttt{\}}
\end{figure}


%
\begin{figure}[H]

\caption{Ode format in Fortran\label{fortran-ode}}

\texttt{\textbf{double precision function}} \texttt{getx0()}

\texttt{\textbf{end}}~\\


\texttt{\textbf{double precision function}} \texttt{getx1()}

\texttt{\textbf{end}}~\\


\texttt{\textbf{integer function}} \texttt{getkind()}

\texttt{\textbf{end}}~\\


\texttt{\textbf{integer function}} \texttt{getnpoints()}

\texttt{\textbf{end}}~\\


\texttt{\textbf{double precision function}} \texttt{getf0()}

\texttt{\textbf{end}}~\\


\texttt{\textbf{double precision function}} \texttt{getf1()}

\texttt{\textbf{end}}~\\


\texttt{\textbf{double precision function}} \texttt{getff0()}

\texttt{\textbf{end}}~\\


\texttt{\textbf{double precision function}} \texttt{ode1ff(x,y,yy)}

\texttt{\textbf{double precision}} \texttt{x,y,yy}

\texttt{\textbf{end}}~\\


\texttt{\textbf{double precision function}} \texttt{ode2ff(x,y,yy,yyy)}

\texttt{\textbf{double precision}} \texttt{x,y,yy,yyy}

\texttt{\textbf{end}}
\end{figure}


The following steps simplify the way to add a new ODE:

\begin{enumerate}
\item Code the ODE either in C++ format or in Fortran format and put the
corresponding file under the directory \texttt{OdeExamples}.
\item Add the name of the ODE in the text file \texttt{odelist} under the
directory named \texttt{OdeExamples}.
\item Execute the script \texttt{makeodes.sh} in the same directory or run
the command \texttt{make} in the main directory of the distribution.
\end{enumerate}
The steps above will create a shared library for the equation to be
solved under the directory \texttt{OdeExamples}. Usually, in UNIX
systems the shared libraries have the extension .so to their names.


\subsection{Format for SODEs}

In figures \ref{sode-c++} and \ref{sode-fortran} we demonstrate
the formulation of SODE's in c++ and in Fortran programming languages
correspondingly. The functions used in those formulations have the
following meanings:

\begin{enumerate}
\item \texttt{getx0()}: Returns the left boundary, $x_{0}$.
\item \texttt{getx1()}: Returns the right boundary, $x_{1}$.
\item \texttt{getnode()}: Returns the number of ODE's in the system (system
size).
\item \texttt{getnpoints()}: Returns the number of the training points for
the system of trial solutions.
\item \texttt{systemfun(node,x,y,yy)} : For the SODE case, the aim of the
\texttt{DeSolver} is to minimize the function $\mbox{systemfun}(n,x,Y,Y')$
for values of $x$ in the range $[x_{0},x_{1}]$, where $n$ is the
total number of equations in the system, $Y$ is a vector with elements
the $n$ functions that are produced by the grammatical evolution
evaluated at $x$ and $Y'$ is a vector with elements the first derivative
of these $n$ equations evaluated at $x$. The argument node represents
the $n$, the double precision array y stands for the vector $Y$
and similar the double precision array yy represents the vector $Y'$. 
\item s\texttt{ystemf0(node,f0)}: The argument node stands for the number
of differential equations in the system and the double precision array
f0 with node elements represents the vector holding the boundary conditions
for each equation in the system.
\end{enumerate}
%
\begin{figure}[H]

\caption{Format for SODEs in C++\label{sode-c++}}

\texttt{\textbf{extern}} \texttt{\char`\"{}C\char`\"{} \{}

\texttt{\textbf{double}} \texttt{getx0() }

\texttt{\{ }

\texttt{\}}~\\


\texttt{\textbf{double}} \texttt{getx1() }

\texttt{\{ }

\texttt{\}}~\\


\texttt{\textbf{int}} \texttt{getnode() }

\texttt{\{ }

\texttt{\}}~\\


\texttt{\textbf{int}} \texttt{getnpoints() }

\texttt{\{ }

\texttt{\}}~\\


\texttt{\textbf{double}} \texttt{systemfun(}\texttt{\textbf{int}}
\texttt{node,} \texttt{\textbf{double}} \texttt{x,} \texttt{\textbf{double}}
\texttt{{*}y,} \texttt{\textbf{double}} \texttt{{*}yy) }

\texttt{\{}

\texttt{\}}~\\


\texttt{\textbf{void}} \texttt{systemf0(}\texttt{\textbf{int}} \texttt{node,}\texttt{\textbf{double}}
\texttt{{*}f0) }

\texttt{\{ }

\texttt{\}}

\texttt{\}} 
\end{figure}


%
\begin{figure}[H]

\caption{Format for SODEs in Fortran\label{sode-fortran}}

\texttt{\textbf{double precision function}} \texttt{getx0() }

\texttt{\textbf{end}}~\\


\texttt{\textbf{double precision function}} \texttt{getx1() }

\texttt{\textbf{end}}~\\


\texttt{\textbf{integer function}} \texttt{getnode() }

\texttt{\textbf{end}}~\\


\texttt{\textbf{integer function}} \texttt{getnpoints() }

\texttt{\textbf{end}}~\\


\texttt{\textbf{double precision function}} \texttt{systemfun(node,x,y,yy) }

\texttt{\textbf{integer}} \texttt{node }

\texttt{\textbf{double precision}} \texttt{x }

\texttt{\textbf{double precision}} \texttt{y(node) }

\texttt{\textbf{double precision}} \texttt{yy(node) }

\texttt{\textbf{end}}~\\


\texttt{\textbf{integer function}} \texttt{systemf0(node,f0) }

\texttt{\textbf{integer}} \texttt{node }

\texttt{\textbf{double precision}} \texttt{f0(node) }

\texttt{\textbf{end }}
\end{figure}


Like the case of ODEs a similar procedure must be followed in order
to add a new SODE in the system. The steps for this procedure has
as follows:

\begin{enumerate}
\item Code the SODE either in C++ format or in Fortran format and put the
corresponding file under the directory \texttt{SOdeExamples}.
\item Add the name of the SODE in the text file \texttt{sodelist} under
the directory named \texttt{SOdeExamples}.
\item Execute the script \texttt{makesodes.sh} in the same directory or
run the command \texttt{make} in the main directory of the distribution.
\end{enumerate}
The steps above will create a shared library for the equation to be
solved under the directory \texttt{SOdeExamples}. 


\subsection{PDE format}

The system is capable of solving elliptic PDEs in two dimensions in
a box $[x_{0},x_{1}]\times[y_{0},y_{1}]$ with the Dirichlet boundary
conditions $\Psi(x_{0},y)=f_{0}(y)$, $\Psi(x_{1},y)=f_{1}(y)$, $\Psi(x,y_{0})=g_{0}(x)$
and $\Psi(x,y_{1})=g_{1}(x)$. In figures \ref{pde-c++} and \ref{Fortran-PDE}
we can see the formulation of PDE's in c++ and Fortran programming
languages. The presented functions have the following representation:

\begin{enumerate}
\item \texttt{getx0()}: Returns the left boundary $x_{0}$.
\item \texttt{getx1()}: Returns the right boundary $x_{1}$.
\item \texttt{gety0()}: Returns the left boundary $y_{0}$.
\item \texttt{gety1()}: Returns the right boundary $y_{1}$.
\item \texttt{getnpoints()}: Returns the amount of interior training points
for the PDE. 
\item \texttt{getbpoints()}: Returns the amount of training points across
each boundary of the PDE.
\item \texttt{f0(y)}: Returns the boundary condition $f_{0}(y)$ across
$x=x_{0}$.
\item \texttt{f1(y)}: Returns the boundary condition $f_{1}(y)$ across
$x=x_{1}.$
\item \texttt{g0(x)}: Returns the boundary condition $g_{0}(x)$ across
$y=y_{0}$.
\item \texttt{g1(x)}: The function returns the boundary condition $g_{1}(x)$
across $y=y_{1}$.
\item \texttt{pde(x,y,v,x1,y1,x2,y2)}: For the PDE case \texttt{DeSolver}
minimizes the function $\mbox{pde}\left(x,y,M(x,y),\frac{\partial M(x,y)}{\partial x},\frac{\partial M(x,y)}{\partial y},\frac{\partial^{2}M(x,y)}{\partial x^{2}},\frac{\partial^{2}M(x,y)}{\partial y^{2}}\right)$,
where $x\in[x_{0},x_{1}]$ and $y\in[y_{0},y_{1}].$ The function
$M(x,y)$ arises from the grammatical evolution and corresponds to
the v argument. The argument x1 corresponds to the first derivative
of $M(x,y)$ with respect to $x$, y1 corresponds to the first derivative
of $M(x,y)$ with respect to y, x2 corresponds to the second derivative
of $M(x,y)$ with respect to x and the y2 corresponds to the second
derivative of $M(x,y)$ with respect to y.
\end{enumerate}
%
\begin{figure}[H]

\caption{Format for PDEs in C++\label{pde-c++}}

\texttt{\textbf{extern}} \texttt{\char`\"{}C\char`\"{} \{}

\texttt{\textbf{double}} \texttt{getx0() }

\texttt{\{ }

\texttt{\}}~\\


\texttt{\textbf{double}} \texttt{getx1() }

\texttt{\{}

\texttt{\}}~\\


\texttt{\textbf{double}} \texttt{gety0() }

\texttt{\{ }

\texttt{\}}~\\


\texttt{\textbf{double}} \texttt{gety1() }

\texttt{\{}

\texttt{\}}~\\


\texttt{\textbf{int}} \texttt{getnpoints() }

\texttt{\{ }

\texttt{\}}~\\


\texttt{\textbf{int}} \texttt{getbpoints() }

\texttt{\{ }

\texttt{\}}~\\


\texttt{\textbf{double}} \texttt{f0(}\texttt{\textbf{double}} \texttt{y) }

\texttt{\{}

\texttt{\}}~\\


\texttt{\textbf{double}} \texttt{f1(}\texttt{\textbf{double}} \texttt{y) }

\texttt{\{ }

\texttt{\}}~\\


\texttt{\textbf{double}} \texttt{g0(}\texttt{\textbf{double}} \texttt{x) }

\texttt{\{ }

\texttt{\} }~\\


\texttt{\textbf{double}} \texttt{g1(}\texttt{\textbf{double}} \texttt{x) }

\texttt{\{}

\texttt{\}}~\\


\texttt{\textbf{double}} \texttt{pde(}\texttt{\textbf{double}} \texttt{x,}
\texttt{\textbf{double}} \texttt{y,} \texttt{\textbf{double}} \texttt{v,}
\texttt{\textbf{double}} \texttt{x1,} \texttt{\textbf{double}} \texttt{y1,}

\texttt{\textbf{$\quad\quad\quad\quad$double}} \texttt{x2,} \texttt{\textbf{double}}
\texttt{y2) }

\texttt{\{ }

\texttt{\}}

\texttt{\}} 
\end{figure}


%
\begin{figure}[H]

\caption{Format for PDEs in Fortran\label{Fortran-PDE}}

\texttt{\textbf{double precision function}} \texttt{getx0() }

\texttt{\textbf{end}}~\\


\texttt{\textbf{double precision function}} \texttt{getx1() }

\texttt{\textbf{end}}~\\


\texttt{\textbf{double precision function}} \texttt{gety0() }

\texttt{\textbf{end}}~\\


\texttt{\textbf{double precision function}} \texttt{gety1() }

\texttt{\textbf{end}}~\\


\texttt{\textbf{integer function}} \texttt{getnpoints() }

\texttt{\textbf{end}}~\\


\texttt{\textbf{integer function}} \texttt{getbpoints() }

\texttt{\textbf{end}}~\\


\texttt{\textbf{double precision function}} \texttt{f0(y) }

\texttt{\textbf{double precision}} \texttt{y }

\texttt{\textbf{end}}~\\


\texttt{\textbf{double precision function}} \texttt{f1(y) }

\texttt{\textbf{double precision}} \texttt{y }

\texttt{\textbf{end}}~\\


\texttt{\textbf{double precision function}} \texttt{g0(x) }

\texttt{\textbf{double precision}} \texttt{x }

\texttt{\textbf{end}}~\\


\texttt{\textbf{double precision function}} \texttt{g1(x) }

\texttt{\textbf{double precision}} \texttt{x }

\texttt{\textbf{end}}~\\


\texttt{\textbf{double precision function}} \texttt{pde(x,y,v,x1,y1,x2,y2) }

\texttt{\textbf{double precision}} \texttt{x,y,v,x1,y1,x2,y2 }

\texttt{\textbf{end}} \texttt{}
\end{figure}


Similarly to the case of ODEs and SODEs, the following procedure must
be followed to add a new PDE in the system:

\begin{enumerate}
\item Code the PDE either in C++ format or in Fortran format and put the
corresponding file under the directory \texttt{PdeExamples}.
\item Add the name of the PDE in the text file \texttt{pdelist} under the
directory named \texttt{PdeExamples}.
\item Execute the script \texttt{makepdes.sh} in the same directory or run
the command \texttt{make} in the main directory of the distribution.
\end{enumerate}
The steps above will create a shared library for the equation to be
solved under the directory \texttt{PdeExamples}. 


\section{Test Runs}


\subsection{Ode example }

Suppose that we want to solve the ordinary differential equation \[
y''y'=-\frac{4}{x^{3}}\]
with $y(1)=0,\  y'(1)=2$ and $x\in[1,2]$. The exact solution is
$y=\log(x^{2})$. In figure \ref{example-c++} we see the formulation
for this ODE expressed in the C++ programming language.%
\begin{figure}[H]

\caption{Codification of $y''y'=-\frac{4}{x^{3}}$ in c++\label{example-c++}}

\texttt{\# include <math.h> }

\texttt{\textbf{extern}} \texttt{\char`\"{}C\char`\"{} \{}

\texttt{\textbf{double}} \texttt{getx0() \{} \texttt{\textbf{return}}
\texttt{1.0; \}}

\texttt{\textbf{double}} \texttt{getx1() \{} \texttt{\textbf{return}}
\texttt{2.0; \}}

\texttt{\textbf{int}} \texttt{getkind() \{} \texttt{\textbf{return}}
\texttt{2; \}}

\texttt{\textbf{int}} \texttt{getnpoints() \{} \texttt{\textbf{return}}
\texttt{10; \}}

\texttt{\textbf{double}} \texttt{getf0() \{} \texttt{\textbf{return}}
\texttt{0.0; \}}

\texttt{\textbf{double}} \texttt{getff0() \{} \texttt{\textbf{return}}
\texttt{2.0; \}}

\texttt{\textbf{double}} \texttt{getf1() \{} \texttt{\textbf{return}}
\texttt{0.0;\}}

\texttt{\textbf{double}} \texttt{ode1ff(}\texttt{\textbf{double}}
\texttt{x,} \texttt{\textbf{double}} \texttt{y,} \texttt{\textbf{double}}
\texttt{yy) \{} \texttt{\textbf{return}} \texttt{0.0; \}}

\texttt{\textbf{double}} \texttt{ode2ff(}\texttt{\textbf{double}}
\texttt{x,} \texttt{\textbf{double}} \texttt{y,} \texttt{\textbf{double}}
\texttt{yy,} \texttt{\textbf{double}} \texttt{yyy)}

\texttt{\{ }

\texttt{\textbf{return}} \texttt{yyy{*}yy+4.0/(x{*}x{*}x); }

\texttt{\}}

\texttt{\} }
\end{figure}
Suppose that the ODE is compiled in a file named \texttt{exode.so}
under the directory \texttt{OdeExamples} of the distribution. Under
the \texttt{bin} directory of the distribution we issue the command:

\begin{lyxcode}
./ode\_cmd~-p~../OdeExamples/exode.so~-r~1000
\end{lyxcode}
In figure \ref{ode_cmd-output} the last line of the output of the
above command is presented. As we can see the ODE is solved exactly
in five generations of the genetic algorithm.

%
\begin{figure}[H]

\caption{Output of the program ode\_cmd \label{ode_cmd-output}}

\texttt{GENERATION:    5         FITNESS: -5.040895611e-31       SOLUTION: log(x{*}x) }
\end{figure}
Note that solving ODEs of the form $y'=f(x)$ corresponds to performing
the indefinite integration $\int f(x)dx$. So, this method may be
used as an integration tool as well.


\subsection{Pde example}

Suppose that we want to find the solution of the partial differential
equation \[
\nabla^{2}\Psi(x,y)=-2\Psi(x,y)\]
with $x\in[0,1],\  y\in[0,1]$ subject to Dirichlet boundary conditions:
$\Psi(0,y)=0,\ \Psi(1,y)=\sin(1)\cos(y),\ \Psi(x,0)=\sin(x),\ \Psi(x,1)=\sin(x)\cos(1)$.
The exact solution is \[
\Psi(x,y)=\sin(x)\cos(y)\]
The representation of the equation in Fortran 77 is given in figure
\ref{Fortran-example}. Suppose that the produced shared library for
that equation is named \texttt{expde.so} under the \texttt{PdeExamples}
directory of the distribution. Under the \texttt{bin} directory we
issue the command:

\begin{lyxcode}
./pde\_cmd~-p~../PdeExamples/expde.so~-r~12345678
\end{lyxcode}
The equation was solved exactly at the $58^{\mbox{th}}$ generation
and the last line of the output of the \texttt{pde\_cmd} program is
given in figure \ref{pde_cmd-output}.

%
\begin{figure}[H]

\caption{Output of the program pde\_cmd \label{pde_cmd-output}}

\texttt{\footnotesize GENERATION: 58 FITNESS: -5.176312884e-32 SOLUTION: (-((-(cos(y){*}sin(x))))) }
\end{figure}


%
\begin{figure}[H]

\caption{Codification of $\nabla^{2}\Psi(x,y)=-2\Psi(x,y)$ in Fortran \label{Fortran-example}}

\texttt{\textbf{double precision function}} \texttt{getx0() }

\texttt{getx0=0.0 }

\texttt{\textbf{end}}

\texttt{\textbf{double precision function}} \texttt{getx1() }

\texttt{getx1=1.0 }

\texttt{\textbf{end}}

\texttt{\textbf{double precision function}} \texttt{gety0()}

\texttt{gety0=0.0 }

\texttt{\textbf{end}}

\texttt{\textbf{double precision function}} \texttt{gety1() }

\texttt{gety1=1.0 }

\texttt{\textbf{end}}

\texttt{\textbf{integer function}} \texttt{getnpoints() }

\texttt{getnpoints=25 }

\texttt{\textbf{end}}

\texttt{\textbf{integer function}} \texttt{getbpoints() }

\texttt{getbpoints=50 }

\texttt{\textbf{end}}

\texttt{\textbf{double precision function}} \texttt{f0(y) }

\texttt{\textbf{double precision}} \texttt{y }

\texttt{f0=0.0 }

\texttt{\textbf{end}}

\texttt{\textbf{double precision function}} \texttt{f1(y) }

\texttt{\textbf{double precision}} \texttt{y }

\texttt{f1=}\texttt{\textbf{sin}}\texttt{(1.D0){*}}\texttt{\textbf{cos}}\texttt{(y) }

\texttt{\textbf{end}}

\texttt{\textbf{double precision function}} \texttt{g0(x) }

\texttt{\textbf{double precision}} \texttt{x }

\texttt{g0=}\texttt{\textbf{sin}}\texttt{(x) }

\texttt{\textbf{end}}

\texttt{\textbf{double precision function}} \texttt{g1(x) }

\texttt{\textbf{double precision}} \texttt{x }

\texttt{g1=}\texttt{\textbf{sin}}\texttt{(x){*}}\texttt{\textbf{cos}}\texttt{(1.D0) }

\texttt{\textbf{end}}

\texttt{\textbf{double precision function}} \texttt{pde(x,y,v,x1,y1,x2,y2) }

\texttt{\textbf{double precision}} \texttt{x,y,v,x1,y1,x2,y2 }

\texttt{pde=x2+y2+2.D0{*}v }

\texttt{\textbf{end }}
\end{figure}


\begin{thebibliography}{1}
\bibitem{key-44}I. G. Tsoulos and I. E. Lagaris, {}``Solving Differential Equations
with Grammatical Evolution'', Technical Report 13-2004, Department
of Computer Science, University of Ioannina, Greece, 2004.
\bibitem{key-38}J. R. Koza, Genetic Programming: On the programming of Computer by
Means of Natural Selection. MIT Press: Cambridge, MA, 1992.
\bibitem{key-39}M. O'Neill and C. Ryan, {}``Under the hood of grammatical evolution,''
In Wolfgang Banzhaf, Jason Daida, Agoston E. Eiben, Max H. Garzon
Vasant Honavar, Mark Jakiela, and Robert E. Smith (eds.), Proceedings
of the Genetic and Evolutionary Computation Conference, vol. 2, Orlando,
Florida, USA, 13-17 July 1999, Morgan Kaufmann, pp. 1143-1148, 1999.
\bibitem{key-40}M. O'Neill and C. Ryan, {}``Evolving Multi-Line Compilable C Programs,''
In Riccardo Poli, Peter Nordin, William B. Langdon, and Terence C.
Fogarty (eds.), Proceedings of EuroGP'99, volume 1598 of LNCS, Goteborg,
Sweden, 26-27 May 1999. Springer-Verlag, pp. 83-92, 1999.
\bibitem{key-41}M. O'Neill and C. Ryan, Grammatical Evolution: Evolutionary Automatic
Programming in a Arbitrary Language, volume 4 of Genetic programming.
Kluwer Academic Publishers, 2003. 
\bibitem{key-42}Tmake: available from http://tmake.sourceforge.net
\bibitem{key-43}J. Nieminen and J. Yliluoma, {}``Function Parser for C++, v2.7'',
available from http://www.students.tut.fi/$\tilde{\ }$warp/FunctionParser/.\end{thebibliography}


\end{document}
